\chapter{Introduction and concept}

\section{Device}
The prototype is built around two ESP32 DevKit boards and two laptops running
the MQTT broker and visualization tools. The goal is to monitor thermal conditions remotely
with a robust encryption system

\subsection*{Hardware components}
\begin{figure}[H]
    \centering
    \includegraphics[width=0.45\textwidth]{Report/Pic/ESP32_schematic.png}
    \caption{ESP32 schematic}
\end{figure}


\begin{itemize}
    \item \textbf{Publisher ESP32}  
    This board collects the environmental data in the room under test. And It sends the information to the MQTT server with encryption. It is connected to:
    \begin{itemize}
        \item a DHT11 temperature and humidity sensor,
        \item a push button used later for entering the Morse code,
        \item two LEDs that provide simple feedback about the system state
              (Wi‑Fi/MQTT connection, data transmission).
    \end{itemize}

    \item \textbf{Subscriber ESP32}  
    This second board plays the role of a remote command center. It receives the encrypted measurements from the MQTT broker, verifies the integrity and authenticity of the messages, and decrypts the payload. The same board also runs a small web server that exposes the current temperature and humidity to any device on the local network.

    \item \textbf{Computers running the MQTT broker}  
    Laptops are used to host the Mosquitto MQTT brokers.
    Both ESP32 boards connect to this broker over Wi‑Fi through a local access point (a smartphone hotspot in our case). The computer can also run additional tools for data visualisation, such as an MQTT dashboard or plotting software.
\end{itemize}

\subsection*{Roles of the devices}
The \textbf{Publisher ESP32} works as a sensor node installed in the room. It periodically reads the DHT11 sensor, builds a JSON payload, encrypts this payload and sends it to the MQTT broker.
The \textbf{Subscriber ESP32} works as the monitored unit of the room. It subscribes to the encrypted topics, checks the HMAC, detects replay attacks using a counter, decrypts the data and republishes a plain version for visualisation. In this way the security mechanisms can be tested without modifying the MQTT broker itself.


\section{Concept}

This is what I wrote in the Threat model section  - you can take it as reference to explain the concept more in details with maybe a figure.

\textit{In the first chapter, we discussed about the project concept where one ESP32 is used as a data publisher mimicking the data collection of a room
we would like the thermal conditions to be controlled via a command center, which is the alias for our second ESP32. \\
The acquised data are sent from our ESP32 "publisher" to a MQTT server, installed on our machine for the example.  The second ESP32 is subscribed
to the MQTT server, receiving the published data and in charge of monitoring - controlling the room conditions.  Those data are displayed
on another MQTT server able to plot data evolution.}







