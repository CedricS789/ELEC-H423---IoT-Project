\chapter{Introduction and concept}

\section{Device}
The prototype is built around two ESP32 DevKit boards and two laptops running
the MQTT broker and visualization tools. The goal is to monitor thermal conditions remotely
with a robust encryption system

\subsection*{Hardware components}
\begin{figure}[H]
    \centering
    \includegraphics[width=0.45\textwidth]{../pic/ESP32_schematic.png}
    \caption{ESP32 schematic}
\end{figure}


\begin{itemize}
    \item \textbf{Publisher ESP32}  
    This board collects the environmental data in the room under test. And It sends the information to the MQTT server with encryption. It is connected to:
    \begin{itemize}
        \item a DHT11 temperature and humidity sensor,
        \item a push button used later for entering the Morse code,
        \item two LEDs that provide simple feedback about the system state
              (Wi‑Fi/MQTT connection, data transmission).
    \end{itemize}

    \item \textbf{Subscriber ESP32}  
    This second board plays the role of a remote command center. It receives the encrypted measurements from the MQTT broker, verifies the integrity and authenticity of the messages, and decrypts the payload. The same board also runs a small web server that exposes the current temperature and humidity to any device on the local network.

    \item \textbf{Computers running the MQTT broker}  
    Laptops are used to host the Mosquitto MQTT brokers.
    Both ESP32 boards connect to this broker over Wi‑Fi through a local access point (a smartphone hotspot in our case). The computer can also run additional tools for data visualisation, such as an MQTT dashboard or plotting software.
\end{itemize}

\subsection*{Roles of the devices}
The \textbf{Publisher ESP32} works as a sensor node installed in the room. It periodically reads the DHT11 sensor, builds a JSON payload, encrypts this payload and sends it to the MQTT broker.
The \textbf{Subscriber ESP32} works as the monitored unit of the room. It subscribes to the encrypted topics, checks the HMAC, detects replay attacks using a counter, decrypts the data and republishes a plain version for visualisation. In this way the security mechanisms can be tested without modifying the MQTT broker itself.


\section{Concept}

The project implements a secure remote monitoring system for temperature and humidity, 
using two ESP32 microcontrollers communicating over MQTT. 
One ESP32 acts as a Publisher, located in the environment to be monitored. 
It collects data from a DHT11 sensor, encrypts the measurements, and transmits them to an MQTT broker.
The second ESP32 acts as a Subscriber. 
It receives the encrypted data, authenticates the source, 
decrypts the payload, and displays the information via a local web server and a visualization dashboard.

To ensure the security of the communication in an open wireless environment, the system implements a multi-layered security scheme. 
Maximizing confidentiality is achieved by ensuring data cannot be read by unauthorized parties using AES-128 encryption. 
We maintain integrity and authentication by verifying that data has not been tampered with and originates from the legitimate publisher using HMAC-SHA256. 
Repudiation protection ensures a definitive record of sequential messages by tracking a monotonically increasing sequence counter. 
For key agreement, we use Morse code, to derive session keys, 
which avoids the vulnerability of having hardcoded keys in the firmware source code. 
Finally, network obfuscation is implemented through a port knocking sequence to complicate traffic analysis.

