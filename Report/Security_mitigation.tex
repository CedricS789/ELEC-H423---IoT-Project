\chapter{Security Mitigation}

\textbf{Discuss about} 

\begin{enumerate}
    \item Hardware being crypted (assumption)
    \item Key being encrypted with usage of morse code through PBDKF2 ( Additional feature !!) and saved in flash
    \item Decryption of the key with morse code before sending payload to get back the plain key 
    \item Encryption of paylaod with AES-128 bits encryption (low ressources demand)
    \item Data sent to MQTT server (assumed to have a secured hardware and files secured too in the OS)
    \item HMAC generation for ensuring authentification (at MQTT server - assumption) at the surbscriber
    \item Replay detection verification with counter (at MQTT server - assumption) at the surbscriber
    \item Port knocking implementation to complexify the sniffing (changing port)
\end{enumerate}



\section{Security scheme}

Figure ~\ref{fig:Encrypted_key}
\begin{figure}[H]
    \centering
    \includegraphics[width=0.7\textwidth]{../pic/Encrypted_key.PNG}
    \caption{Encrypted key generation}
    \label{fig:Encrypted_key}
\end{figure}

Figure ~\ref{fig:Security_pipeline}
\begin{figure}[H]
    \centering
    \includegraphics[width=0.7\textwidth]{../pic/Security_pipeline.PNG}
    \caption{Security strategy integration in the project data flow}
    \label{fig:Security_pipeline}
\end{figure}

\section{Results}

In thi section, we will review the results obtained during the implementation of the thermal regulation concept - previously discussed - with
 two ESP32 communicating through a MQTT server in a fully securised environment (authentification, confidentiality and integrity).

We will proceed step by step by confirming the implementation of :

\begin{enumerate}
    \item The morse code acquisition with debouncing function for security
    \item PBKDF2 key generation and decryption of the encrypted key in the flash memory
    \item Encryption of the payload with the decrypted key
    \item Decryption of the payload at the subscriber
    \item Replay detection verification
    \item Port knocking verification
\end{enumerate}

\subsection{Morse Code}
The morse code acquisition passes by the implementation of a look-up table translated unique succession of \textbf{dots} and \textbf{dashes} 
to an alphabetic letter. These \textbf{dots} and \textbf{dashes} being generated depending on the duration when the push button has been pressed.
Just to ensure the translation code was correctly implemented, the code compares the morse code encoded and the one we should use to decrypt the
key stored in the flash. Of course, this verification is not done in practice as it would mean the morse code is stored in the flash memory which
could be an additional break in our security strategy - morse code is supposed to be kept secret and shared verbally.

Figure ~\ref{fig:Morse_Code_check} shows the results of the last letter encoding (and the  the verification of the morse code).

\begin{figure}[H]
    \centering
    \includegraphics[width=0.3\textwidth]{../pic/Morse_Code_check.PNG}
    \caption{Morse code implementation Result}
    \label{fig:Morse_Code_check}
\end{figure}


\subsection{PBKDF2 encryption}

The following security phase is the decryption of the encrypted key saved in the non-volatile memory through the PBKDF2 encryption and AES decryption.
The figure ~\ref{fig:PBKDF2_result} shows the following results :

\begin{enumerate}
    \item Morse code encoded with the push button
    \item Derived 16-bytes key generated by the PBKDF2 algorithm (Morse, "Salt", HMAC 256, 1000 iterations)
    \item Encrypted 32-bytes key saved in the flash - $\text{Key}_\text{enc}= \text{AES}(\text{IV code}, \text{Key}_\text{plain})$
    \item Decryption of the key - $\text{Key}_\text{plain}= \text{Dec}(\text{IV code}, \text{Key}_\text{enc})$
    \item $\text{Key}_\text{plain} =  0\text{x}54,0\text{x}68,0\text{x}61,0\text{x}74,0\text{x}73,0\text{x}20,0\text{x}6d,0\text{x}79,0\text{x}20,0\text{x}4b,0\text{x}75,0\text{x}6e,0\text{x}67,0\text{x}20,0\text{x}46,0\text{x}75$ - \textbf{Thats my Kung Fu} in 16 ASCII
\end{enumerate}

\begin{figure}[H]
    \centering
    \includegraphics[width=0.6\textwidth]{../pic/PBKDF2_result.PNG}
    \caption{PBKDRF2 encryption verification}
    \label{fig:PBKDF2_result}
\end{figure}

\subsection{Encrypted data publishment}

Once the plain key ($\text{Key}_\text{plain}$) has been generated, the thermal data can be encrypted with AES-128 bits encryption.
The figure ~\ref{fig:Encrypted_data_publish} shows the details fo the encryption functionality and the results of the payload encryption.

\begin{enumerate}
    \item Thermal conditions gathered in a java script object notation (readable text to send structured data)
    \item IV and plain key shown for transparency in \textbf{Arduino String}
    \item Hmac and cypher payload generation with AES-128 bits encryption.
    \item Publishment of the data on the MQTT server after client connexion on MQTT server on dedicated port.
\end{enumerate}

\begin{figure}[H]
    \centering
    \includegraphics[width=0.8\textwidth]{../pic/Encrypted_data_publish.PNG}
    \caption{Encrypted data publishment on MQTT server}
    \label{fig:Encrypted_data_publish}
\end{figure}

\subsection{Encrypted data publishment}

Figure ~\ref{fig:Decrypted_data_subscriber}
\begin{figure}[H]
    \centering
    \includegraphics[width=0.8\textwidth]{../pic/Decrypted_data_subscriber.PNG}
    \caption{Decryption verification}
    \label{fig:Decrypted_data_subscriber}
\end{figure}

\subsection{Replay detection}

Figure ~\ref{fig:Replay_detection}
\begin{figure}[H]
    \centering
    \includegraphics[width=0.5\textwidth]{../pic/Replay_detection.PNG}
    \caption{Counter implementation for replay counter strategy}
    \label{fig:Replay_detection}
\end{figure}

\subsection{Port Knocking}

Figure ~\ref{fig:Port_Knocking_visualization}
\begin{figure}[H]
    \centering
    \includegraphics[width=0.8\textwidth]{../pic/Port_Knocking_visualization.PNG}
    \caption{Port switching for sniffing security improvement}
    \label{fig:Port_Knocking_visualization}
\end{figure}

\section{Discussion}

\textbf{Discuss about : }

\begin{enumerate}
    \item the assumption the MQTT server and the ESP32 flashs were secured - hardware level - trust zone area !
    \item That the port knocking can work only if the server switches from one listening port to another listening port in synchronization with the ESP32
    \item Explain that attacks like low-level instructions counting can be used to understand the encryption we used (last year course with Tobias)?
    \item Explain the maybe AES 128-bits encryption is a bit light for sensitive data ? I do not know but maybe be good to check
\end{enumerate}



\section{Conclusion}

