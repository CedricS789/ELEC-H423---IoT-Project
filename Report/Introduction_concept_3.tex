\chapter{Introduction and concept}

\section{Device}
The Wireless and Mobile Network project for the acadmeic year 2025-2026 is built around two ESP32 DevKit boards, electronic sensor and actuators and two laptops running
MQTT brokers and visualization tools. The objective of the project is to develop and simulate a concept focused on secure communication in an IoT environment.
 
\subsection*{Hardware components}
\begin{figure}[H]
    \centering
    \includegraphics[width=0.4\textwidth]{../pic/ESP32_schematic.png}
    \caption{ESP32 schematic}
\end{figure}

The details of the hardwares at disposal are enumerated below:

\begin{itemize}
    \item \textbf{ESP32 boards}
    The ESP32 is a low-power, energy-efficient microcontroller containing multiples features as 
    \begin{itemize}
        \item 4MB flash memory
        \item 520KB RAM
        \item Dual-core 32-bit processor
        \item Wifi - bluetooth connectivity
        \item cryptographic hardware acceleration 
        \item Low power hardware (PMU, low power core processor)
        \item Pheripheral interfaces (GIOPs, DAC, ADC, UART)
    \end{itemize}
    \item \textbf{Sensor and actuators}   
    \begin{itemize}
        \item a DHT11 temperature and humidity sensor,
        \item a push button used later for entering the Morse code,
        \item two LEDs that provide simple feedback about the system state
              (Wi‑Fi/MQTT connection, data transmission).
    \end{itemize}

    \item \textbf{Computers running the MQTT broker}  
    Personal laptops are used to host the Mosquitto MQTT brokers.
    Both ESP32 boards connect to this broker over Wi‑Fi through a local access point (a smartphone hotspot in our case). The computer can also run additional tools for data visualisation, such as an MQTT dashboard or plotting software.
\end{itemize}

%\subsection*{Roles of the devices}
%The \textbf{Publisher ESP32} works as a sensor node installed in the room. It periodically reads the DHT11 sensor, builds a JSON payload, encrypts this payload and sends it to the MQTT broker.
%The \textbf{Subscriber ESP32} works as the monitored unit of the room. It subscribes to the encrypted topics, checks the HMAC, detects replay attacks using a counter, decrypts the data and republishes a plain version for visualisation. In this way the security mechanisms can be tested without modifying the MQTT broker itself.

\section{Concept}

The project simulates a highly sensitive, climate-controlled room, such as a medical storage area, where temperature and humidity 
must be continuously monitored and regulated. It uses two ESP32 microcontrollers communicating over MQTT through publish-subscribe 
relation and implementing a secure remote monitoring and control system.  Based on the received measurements, the control center
 can issue control signals, e.g. adjusting an AC unit, to maintain the required environmental conditions.\\
MQTT protocol is a lightweight internet protocol which relies to message queuing service for communication from machine to machine. It runs 
over lossless transport layer, in our case TCP.  Two types of entities consitute a MQTT communication - the client(s), publishing or receiving
messages and the broker, the server collecting the packets and redistributing to the messages to subscribed clients.  The MQTT protocol
designed around the publish-subscribe communication with small overhead and payload size fits well to IoT devices - devices limited
 in bandwidth, located in environment with unstable connection and designe for low power consumption.\\

For the project, one ESP32 acts as a \textbf{publisher}, located in the environment to be monitored. 
It collects data from a DHT11 sensor and transmits them to an MQTT broker. The second ESP32 acts as a \textbf{subscriber}.
The drawback of MQTT is that its packets are sent in plaintext by default, so additional measures are required to secure our high sensitive
 communication. Considering that TLS or certificatation authentification are not allowed, the whole idea will be to design a multi-layered
 security strategy ensuring confidentiality, authentication, and integrity of all exchanged data.\\

%It receives the encrypted data, authenticates the source, 
%decrypts the payload, and displays the information via a local web server and a visualization dashboard.
%Based on the received measurements, the control center can issue control signals, e.g. adjusting an 
%AC unit, to maintain the required environmental conditions.

%To ensure the security of the communication in an open wireless environment, the system implements a multi-layered security scheme. 
%Maximizing confidentiality is achieved by ensuring data cannot be read by unauthorized parties using AES-128 encryption. 
%We maintain integrity and authentication by verifying that data has not been tampered with and originates from the legitimate publisher using HMAC-SHA256. 
%Repudiation protection ensures a definitive record of sequential messages by tracking a monotonically increasing sequence counter. 
%For key agreement, we use Morse code, to derive session keys, 
%which avoids the vulnerability of having hardcoded keys in the firmware source code. 
%Finally, network obfuscation is implemented through a port knocking sequence to complicate traffic analysis.

