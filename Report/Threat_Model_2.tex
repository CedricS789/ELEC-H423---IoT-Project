\chapter{Threat Model}

\section{Data flow diagram}

In the first chapter, we discussed about the project concept where one ESP32 is used as a data publisher mimicking the data collection of a room
we would like the thermal conditions to be controlled via a command center, which is the alias for our second ESP32. \\
The acquised data are sent from our ESP32 "publisher" to a MQTT server, installed on our machine for the simulation.  The second ESP32 is subscribed
to the MQTT server, receiving the published data and in charge of monitoring - controlling the room conditions.  Those data are
 displayed on another MQTT server where the data evolution can be plotted with MQTT explorer software.\newline
The data flow is shown on the figure ~\ref{fig:Diagram_flow}

\begin{figure}[H]
    \centering
    \includegraphics[width=0.8\textwidth]{Pic/Diagram_flow.PNG}
    \caption{Data flow diagram}
    \label{fig:Diagram_flow}
\end{figure}

As remainder, the communication being sensitive, it would need to be robust to any types of attacks and mitigate them by integrating
 a security strategy considering the constraints on the hardware ressources.

Thus HTTPS / TLS protocol are not well adapted for our communication and our devices (large payload - heavy on CPU usage - multiple handshake - large memory usage).
 Also key cryptoghraphy containing complex mathematical operation like RSA can be implemented but not suitable for our project (same concerns that TLS).  
Our security strategy focuses on mitigating the relevant network- and protocol-level threats while remaining compatible with the device capabilities and the required decryption speed.
The complete mitigation strategy and the additional security features implemented by our group are presented in the last chapter regarding \textit{Security Mitigation}.

Before discussing about the security strategy, we would need first to identify the security threats our model needs to deal with.  The STRIDE model will
be used in this perspective in the next section.

\begin{figure}[H]
    \centering
    \includegraphics[width=0.8\textwidth]{Pic/Diagram_flow_STRIDE.PNG}
    \caption{STRIDE model verification for project data flow before security mitigations}
    \label{fig:Diagram_flow_STRIDE}
\end{figure}


\section{STRIDE Threat Model}

The STRIDE threat model is an approach used to identify and categorize potential security threats and vulnerabilities in software systems. Developed by Microsoft, the acronym STRIDE represents six main categories of threats. These figures below illustrate all categories, their description and how they are mitigated in the project.\\

\begin{figure}[h]
\includegraphics[width=0.7\textwidth]{Pic/STRIDE.PNG}
\caption{STRIDE model and threats localization}
\label{fig:STRIDE}
\end{figure}

\begin{table}[h!]
    \centering
    \caption{STRIDE Threat Model and Security Mitigations for ESP32-MQTT Setup}
    \label{tab:stride_mitigation}
    \begin{tabular}{>{\raggedright\arraybackslash}p{3cm} >{\raggedright\arraybackslash}p{3.5cm} >{\raggedright\arraybackslash}p{3.5cm} >{\raggedright\arraybackslash}p{6.5cm}}
        \toprule
        \textbf{STRIDE Category} & \textbf{Threat Description} & \textbf{Attacked Component/Data Flow} & \textbf{Security Mitigation (Protocol)} \\
        \midrule
        \textbf{Spoofing} (Authentication) & An attacker impersonates a legitimate ESP32 device or the MQTT Server to inject false data or commands. & ESP32 $\leftrightarrow$ Server Communication & \textbf{Authentication} using a \textbf{Message Authentication Code (MAC)}, such as HMAC-SHA256, to verify the sender's identity with a shared secret key derived from PBKDF2. \\
        \midrule
        \textbf{Tampering - communication channel} (Integrity) & Data (sensor readings) is intercepted and modified while in transit over the network & Sensor Data $\to$ Server, Control Commands & \textbf{Integrity} enforced by the MAC (HMAC-SHA256). The receiver recalculates and compares the MAC to detect any alteration. \\
        \midrule
        \textbf{Tampering - hardware} (Integrity - Authentication) & Security algorithms or keys saved in the hardware are altered & Clients and brocker hardware. & \textbf{Authentification and Integrity} Authentification ensured by the HMAC verification at the receiver - in the contrary the integrity of the message can only be ensured by securing the hardware (assumption). \\
        \midrule
        \textbf{Information Disclosure} (Confidentiality) & An attacker eavesdrops on the network and reads sensitive data (Temperature,Humidity...). & All Communication over the Network & \textbf{Encryption} of the MQTT payload using AES-128. An additional layer is provided by Port Knocking to make sniffing and channel monitoring harder. \\
        \midrule
        \textbf{Denial of Service (DoS)} & An attacker floods the ESP32 or MQTT server with excessive traffic or requests, consuming resources. & ESP32 or Server Resources & \textbf{Rate-limiting} Controls of the packets rate sending at the publisher and \textit{quiet mode} state at receiver.  controls on the MQTT broker (assumption of quiet mode and rejection with HMAC). Lastly, the use of Port Knocking to obscure the communication channel, reducing the window for targeted attacks. \\
        \midrule
        \textbf{Repudiation} (Non-Repudiation) & A legitimate device or the server denies having sent a critical or malicious message. & All Communication & The HMAC proves the message's origin from the authentic sender, and the counter ensures a definitive record of sequential messages, preventing the sender from denying transmission. \\
        \midrule
        \textbf{Elevation of Privilege} (Authorization) & A compromised device gains access to MQTT topics or capabilities it should not have permission for. & ESP32 $\leftrightarrow$ Server Communication, MQTT Topic Subscriptions & Strict Authorization controls on the MQTT broker (assumption) - Secured hardwares (assumption) - the requirement to correctly decrypt the secret key via the Morse code sequence before publishing data. \\
        \bottomrule
    \end{tabular}
\end{table}

